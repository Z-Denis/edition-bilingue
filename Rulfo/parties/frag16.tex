\fancyhead[CE]{Fragmento \esno 16}
\fancyhead[CO]{Fragment \frno 16}

\begin{pages}
	\selectlanguage{spanish}
	\begin{Leftside}
		\beginnumbering
		\pstart[\pagestyle{empty}\unexpanded{\section[Fragmento 16/Fragment 16]{Fragmento \esno 16. Pedro Páramo}}]
			Había estrellas fugaces. Las luces en Comala se apagaron.
		\pend
		\pstart
			Entonces el cielo se adueñó de la noche.
		\pend
		\pstart
			El padre Rentería se revolcaba en su cama sin poder dormir:

			\guillemotleft Todo esto que sucede es por mi culpa --se dijo--. El temor de ofender a quienes me sostienen. Porque ésta es la verdad; ellos me dan mi mantenimiento. De los pobres no consigo nada; las oraciones no llenan el estómago. Así ha sido hasta ahora. Y éstas son las consecuencias. Mi culpa. He traicionado a aquellos que me quieren y que me han dado su fe y me buscan para que yo interceda por ellos para con Dios. ¿Pero qué han logrado con su fe? ¿La ganancia del cielo? ¿O la purificación de sus almas? Y para qué purifican su alma, si en el último momento\ldots Todavía tengo frente a mis ojos la mirada de María Dyada, que vino a pedirme salvara a su hermana Eduviges:
		\pend
		\pstart
			\guillemotright --- Ella sirvió siempre a sus semejantes. Les dio todo lo que tuvo. Hasta les dio un hijo, a todos. Y se los puso enfrente para que alguien lo reconociera como suyo; pero nadie lo quiso hacer. Entonces les dijo: ``En ese caso yo soy también su padre, aunque por casualidad haya sido su madre.'' Abusaron de su hospitalidad por esa bondad suya de no querer ofenderlos ni de malquistarse con ninguno.
		\pend
		\pstart
			\guillemotright --- Pero ella se suicidó. Obró contra la mano de Dios.
		\pend
		\pstart
			\guillemotright --- No le quedaba otro camino. Se resolvió a eso también por bondad.
		\pend
		\pstart
			\guillemotright --- Falló a última hora --eso es lo que le dije--. En el último momento.
			¡Tantos bienes acumulados para su salvación, y perderlos así de pronto!
		\pend
		\pstart
			\guillemotright --- Pero si no los perdió. Murió con muchos dolores. Y el dolor\ldots Usted nos ha dicho algo acerca del dolor que ya no recuerdo. Ella se fue por ese dolor. Murió retorcida por la sangre que la ahogaba. Todavía veo sus muecas, y sus muecas eran los más tristes gestos que ha hecho un ser humano.
		\pend
		\pstart
			\guillemotright --- Tal vez rezando mucho.
		\pend
		\pstart
			\guillemotright --- Vamos rezando mucho, padre.
		\pend
		\pstart
			\guillemotright --- Digo tal vez, si acaso, con las misas gregorianas; pero para eso necesitamos pedir ayuda, mandar traer sacerdotes. Y eso cuesta dinero.
		\pend
		\pstart
			\guillemotright Allí estaba frente a mis ojos la mirada de María Dyada, una pobre mujer llena de hijos.
		\pend
		\pstart
			\guillemotright --- No tengo dinero. Eso lo sabe, padre.
		\pend
		\pstart
			\guillemotright --- Dejemos las cosas como están. Esperemos en Dios.
		\pend
		\pstart
			\guillemotright --- Sí, padre.\guillemotright
		\pend
		\pstart
			¿Por qué aquella mirada se volvía valiente ante la resignación? Qué le costaba a él perdonar, cuando era tan fácil decir una palabra o dos, o cien palabras si éstas fueran necesarias para salvar el alma. ¿Qué sabía él del cielo y del infierno? Y sin embargo, él, perdido en un pueblo sin nombre, sabía los que habían merecido el cielo. Había un catálogo. Comenzó a recorrer los santos del panteón católico comenzando por los del día: \guillemotleft Santa Nunilona, virgen y mártir; Abercio, obispo; santas Salomé viuda, Alodia o Elodia y Nulina, vírgenes; Córdula y Donato.\guillemotright Y siguió. Ya iba siendo dominado por el sueño cuando se sentó en la cama: \guillemotleft Estoy repasando una hilera de santos como si estuviera viendo saltar cabras.\guillemotright
		\pend
		\pstart
			Salió fuera y miró el cielo. Llovían estrellas. Lamentó aquello porque
			hubiera querido ver un cielo quieto. Oyó el canto de los gallos. Sintió la
			envoltura de la noche cubriendo la tierra. La tierra, \guillemotleft este valle de lágrimas\guillemotright.
		\pend
		\endnumbering
	\end{Leftside}
	\selectlanguage{french}
	\begin{Rightside}
		\beginnumbering
		\pstart[\pagestyle{empty}\unexpanded{\section[Fragmento 16/Fragment 16]{Fragment \frno 16. Pédro Paramo}}]
			Il y avait des étoiles filantes. Les lumières s'éteignirent à Comala.
		\pend
		\pstart
			Alors le ciel prit possession de la nuit.
		\pend
		\pstart
			Le père Rentéria se retournait dans son lit sans pouvoir dormir :

			\og{} Tout ce qui se passe est de ma faute --\,se dit-il\,--. La crainte d’offenser ceux qui me soutiennent. Parce que voilà la vérité ; ils m’entretiennent. Des pauvres je n’obtiens rien ; les prières ça ne nourrit pas son homme. Il en a été ainsi jusqu’ici. Et voilà les conséquences. Ma faute. J’ai trahi ceux qui m’aiment et qui m’ont donné leur foi et viennent me chercher pour que j’intercède pour eux auprès de Dieu. Mais qu’ont-ils obtenu avec leur foi ? Une place au Ciel ? Ou la purification de leurs âmes ? Et à quoi bon purifier son âme, si au dernier moment\ldots J’ai encore devant les yeux le regard de Maria Dyada, qui est venu me demander de sauver sa sœur Éduviges :
		\pend
		\pstart
			\guillemotright --- Elle a toujours été au service de ses semblables. Leur a donné tout ce qu’elle avait. Même un fils, à tous. Et elle le leur a mis devant les yeux pour que quelqu’un le reconnaisse comme sien ; mais personne n’a voulu le faire. Alors elle a dit : ``Dans ce cas je suis également son père, bien que par hasard j’en aie été la mère.'' Ils ont abusé de son hospitalité à cause de cette bonté qu’elle avait à ne pas vouloir offenser ni se mettre personne à dos.
		\pend
		\pstart
			\guillemotright --- Mais elle s’est suicidée. Elle a œuvré contre la main de Dieu.
		\pend
		\pstart
			\guillemotright --- Elle n’avait pas d’autre issue. Elle s’y est résolue aussi par bonté.
		\pend
		\pstart
			\guillemotright --- Elle a failli à la dernière minute --\,voilà ce que je lui ai dit\,--. Au dernier moment. Tant de vertu accumulée pour son salut et la perdre comme ça, d'un coup !
		\pend
		\pstart
			\guillemotright --- Mais puisqu’elle ne l’a pas perdue. Elle est morte dans d’atroces souffrances. Et la souffrance… Vous nous avez dit quelque-chose à propos de la souffrance que je ne me rappelle plus. Elle est partie par cette souffrance. Morte tordue par le sang qui la noyait. Je vois encore ses grimaces, et ses grimaces étaient les gestes les plus tristes qu’un être humain ait jamais faits.
		\pend
		\pstart
			\guillemotright --- Peut-être redoublant de prières.
		\pend
		\pstart
			\guillemotright --- On ne fait que ça, père.
		\pend
		\pstart
			\guillemotright --- Je dirais bien peut-être, éventuellement, avec les messes grégoriennes ; mais pour cela on doit demander de l’aide, faire venir des prêtres. Et cela a un prix.
		\pend
		\pstart
			\guillemotright Là face à mes yeux se trouvait le regard de Maria Dyada, une pauvre femme pleine d’enfants.
		\pend
		\pstart
			\guillemotright --- Je n’ai pas cet argent. Vous le savez bien, père.
		\pend
		\pstart
			\guillemotright --- Laissons les choses comme elles sont. Comptons sur Dieu.
		\pend
		\pstart
			\guillemotright --- Oui, père. \fg{}
		\pend
		\pstart
			Pourquoi ce regard s’armait-il de courage face à la résignation ? Que lui coûtait-il, à lui, de pardonner, alors qu’il était si simple de dire un mot ou deux, ou cent mots si cela s’avérait nécessaire pour sauver une âme. Que savait-il, lui, du Ciel et de l’Enfer ? Et pourtant, lui, perdu au milieu d’un village sans nom, savait ceux qui avaient mérité le Ciel. Il y avait un catalogue. Il commença à parcourir les saints du panthéon catholique en commençant par ceux du jour : \og{} Sainte Nunilone, vierge et martyre ; Abercius, évêque ; les Saintes Salomé, veuve, Alodie ou Élodie et Nunilo, vierges ; Cordule et Donat. \fg{}Et il continuat. Le sommeil commençait déjà à le dominer lorsqu’il s’assit sur le lit : \og{} Je révise une suite de saints comme je compterais les moutons.\fg{}
		\pend
		\pstart
			Il sortit et regarda le ciel. Il pleuvait des étoiles. Il le regretta parce qu’il aurait voulu voir un ciel immobile. Il entendit les coqs chanter. Il sentit l’enveloppe de la nuit recouvrant la terre. La terre, \og{}cette vallée de larmes\fg{}.
		\pend
		\endnumbering
	\end{Rightside}

\end{pages}
\Pages
