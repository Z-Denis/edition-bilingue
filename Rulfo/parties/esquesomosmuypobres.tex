\fancyhead[CE]{Es que somos muy pobres}
\fancyhead[CO]{C'est qu'on est très pauvres}

\begin{pages}
	\selectlanguage{spanish}
	\begin{Leftside}
		\beginnumbering

		\pstart[\pagestyle{empty}\unexpanded{\section[Es que somos muy pobres/C'est qu'on est très pauvres]{Es que somos muy pobres. El Llano en llamas}}]
			Aquí todo va de mal en peor. La semana pasada se murió mi tía Jacinta, y el sábado, cuando ya la habíamos enterrado y comenzaba a bajársenos la tristeza, comenzó a llover como nunca. A mi papá eso le dio coraje, porque toda la cosecha de cebada estaba asoleándose en el solar. Y el aguacero llegó de repente, en grandes olas de agua, sin darnos tiempo ni siquiera a esconder aunque fuera un manojo; lo único que pudimos hacer, todos los de mi casa, fue estarnos arrimados debajo del tejabán, viendo cómo el agua fría que caía del cielo quemaba aquella cebada amarilla tan recién cortada.
		\pend

		\pstart
			Y apenas ayer, cuando mi hermana Tacha acababa de cumplir doce años, supimos que la vaca que mi papá le regaló para el día de su santo se la había llevado el río.
		\pend

		\pstart
			El río comenzó a crecer hace tres noches, a eso de la madrugada. Yo estaba muy dormido y, sin embargo, el estruendo que traía el río al arrastrarse me hizo despertar en seguida y pegar el brinco de la cama con mi cobija en la mano, como si hubiera creído que se estaba derrumbando el techo de mi casa. Pero después me volví a dormir, porque reconocí el sonido del río y porque ese sonido se fue haciendo igual hasta traerme otra vez el sueño.
		\pend

		\pstart
			Cuando me levanté, la mañana estaba llena de nublazones y parecía que había seguido lloviendo sin parar. Se notaba en que el ruido del río era más fuerte y se oía más cerca. Se olía, como se huele una quemazón, el olor a podrido del agua revuelta.
		\pend

		\pstart
			A la hora en que me fui a asomar, el río ya había perdido sus orillas. Iba subiendo poco a poco por la calle real, y estaba metiéndose a toda prisa en la casa de esa mujer que le dicen \textit{la Tambora}. El chapaleo del agua se oía al entrar por el corral y al salir en grandes chorros por la puerta. \textit{La Tambora} iba y venía caminando por lo que era ya un pedazo de río, echando a la calle sus gallinas para que se fueran a esconder a algún lugar donde no les llegara la corriente.
		\pend

		\pstart
			Y por el otro lado, por donde está el recodo, el río se debía de haber llevado, quién sabe desde cuándo, el tamarindo que estaba en el solar de mi tía Jacinta, porque ahora ya no se ve ningún tamarindo. Era el único que había en el pueblo, y por eso nomás la gente se da cuenta de que la creciente esta que vemos es la más grande de todas las que ha bajado el río en muchos años.
		\pend

		\pstart
			Mi hermana y yo volvimos a ir por la tarde a mirar aquel amontonadero de agua que cada vez se hace más espesa y oscura y que pasa ya muy por encima de donde debe estar el puente. Allí nos estuvimos horas y horas sin cansarnos viendo la cosa aquella. Después nos subimos por la barranca, porque queríamos oír bien lo que decía la gente, pues abajo, junto al río, hay un gran ruidazal y sólo se ven las bocas de muchos que se abren y se cierran y como que quieren decir algo; pero no se oye nada. Por eso nos subimos por la barranca, donde también hay gente mirando el río y contando los perjuicios que ha hecho. Allí fue donde supimos que el río se había llevado a \textit{la Serpentina} la vaca esa que era de mi hermana Tacha porque mi papá se la regaló para el día de su cumpleaños y que tenía una oreja blanca y otra colorada y muy bonitos ojos.
		\pend

		\pstart
			No acabo de saber por qué se le ocurriría a La Serpentina pasar el río este, cuando sabía que no era el mismo río que ella conocía de a diario. \textit{La Serpentina} nunca fue tan atarantada. Lo más seguro es que ha de haber venido dormida para dejarse matar así nomás por nomás. A mí muchas veces me tocó despertarla cuando le abría la puerta del corral porque si no, de su cuenta, allí se hubiera estado el día entero con los ojos cerrados, bien quieta y suspirando, como se oye suspirar a las vacas cuando duermen.
		\pend

		\pstart
			Y aquí ha de haber sucedido eso de que se durmió. Tal vez se le ocurrió despertar al sentir que el agua pesada le golpeaba las costillas. Tal vez entonces se asustó y trató de regresar; pero al volverse se encontró entreverada y acalambrada entre aquella agua negra y dura como tierra corrediza. Tal vez bramó pidiendo que le ayudaran.
		\pend

		\pstart
			Bramó como sólo Dios sabe cómo.
		\pend

		\pstart
			Yo le pregunté a un señor que vio cuando la arrastraba el río si no había visto también al becerrito que andaba con ella. Pero el hombre dijo que no sabía si lo había visto. Sólo dijo que la vaca manchada pasó patas arriba muy cerquita de donde él estaba y que allí dio una voltereta y luego no volvió a ver ni los cuernos ni las patas ni ninguna señal de vaca. Por el río rodaban muchos troncos de árboles con todo y raíces y él estaba muy ocupado en sacar leña, de modo que no podía fijarse si eran animales o troncos los que arrastraba.
		\pend

		\pstart
			Nomás por eso, no sabemos si el becerro está vivo, o si se fue detrás de su madre
			río abajo. Si así fue, que Dios los ampare a los dos.
		\pend

		\pstart
			La apuración que tienen en mi casa es lo que pueda suceder el día de mañana, ahora que mi hermana Tacha se quedó sin nada. Porque mi papá con muchos trabajos había conseguido a \textit{la Serpentina}, desde que era una vaquilla, para dársela a mi hermana, con el fin de que ella tuviera un capitalito y no se fuera a ir de piruja como lo hicieron mis otras dos hermanas, las más grandes.
		\pend

		\pstart
			Según mi papá, ellas se habían echado a perder porque éramos muy pobres en mi casa y ellas eran muy retobadas. Desde chiquillas ya eran rezongonas. Y tan luego que crecieron les dio por andar con hombres de lo peor, que les enseñaron cosas malas. Ellas aprendieron pronto y entendían muy bien los chiflidos, cuando las llamaban a altas horas de la noche. Después salían hasta de día. Iban cada rato por agua al río y a veces, cuando uno menos se lo esperaba, allí estaban en el corral, revolcándose en el suelo, todas encueradas y cada una con un hombre trepado encima.
		\pend

		\pstart
			Entonces mi papá las corrió a las dos. Primero les aguantó todo lo que pudo; pero más tarde ya no pudo aguantarlas más y les dio carrera para la calle. Ellas se fueron para Ayutla o no sé para dónde; pero andan de pirujas.
		\pend

		\pstart
			Por eso le entra la mortificación a mi papá, ahora por la Tacha, que no quiere vaya a resultar como sus otras dos hermanas, al sentir que se quedó muy pobre viendo la falta de su vaca, viendo que ya no va a tener con qué entretenerse mientras le da por crecer y pueda casarse con un hombre bueno, que la pueda querer para siempre. Y eso ahora va a estar difícil. Con la vaca era distinto, pues no hubiera faltado quien se hiciera el ánimo de casarse con ella, sólo por llevarse también aquella vaca tan bonita.
		\pend

		\pstart
			La única esperanza que nos queda es que el becerro esté todavía vivo. Ojalá no se le haya ocurrido pasar el río detrás de su madre. Porque si así fue, mi hermana Tacha está tantito así de retirado de hacerse piruja. Y mamá no quiere.

		\pend

		\pstart
			Mi mamá no sabe por qué Dios la ha castigado tanto al darle unas hijas de ese modo, cuando en su familia, desde su abuela para acá, nunca ha habido gente mala. Todos fueron criados en el temor de Dios y eran muy obedientes y no le cometían irreverencias a nadie. Todos fueron por el estilo. Quién sabe de dónde les vendría a ese par de hijas suyas aquel mal ejemplo. Ella no se acuerda. Le da vueltas a todos sus recuerdos y no ve claro dónde estuvo su mal o el pecado de nacerle una hija tras otra con la misma mala costumbre. No se acuerda. Y cada vez que piensa en ellas, llora y dice: \og{}Que Dios las ampare a las dos.\fg{}

		\pend

		\pstart
			Pero mi papá alega que aquello ya no tiene remedio. La peligrosa es la que queda aquí, la Tacha, que va como palo de ocote crece y crece y que ya tiene unos comienzos de senos que prometen ser como los de sus hermanas: puntiagudos y altos y medio alborotados para llamar la atención.

		\pend

		\pstart
			--- Sí --dice--, le llenará los ojos a cualquiera dondequiera que la vean. Y acabará mal;
			como que estoy viendo que acabará mal. Ésa es la mortificación de mi papá.
		\pend

		\pstart
			Y Tacha llora al sentir que su vaca no volverá porque se la ha matado el río. Está aquí a mi lado, con su vestido color de rosa, mirando el río desde la barranca y sin dejar de llorar. Por su cara corren chorretes de agua sucia como si el río se hubiera metido dentro de ella.
		\pend

		\pstart
			Yo la abrazo tratando de consolarla, pero ella no entiende. Llora con más ganas. De su boca sale un ruido semejante al que se arrastra por las orillas del río, que la hace temblar y sacudirse todita, y, mientras, la creciente sigue subiendo. El sabor a podrido que viene de allá salpica la cara mojada de Tacha y los dos pechitos de ella se mueven de arriba abajo, sin parar, como si de repente comenzaran a hincharse para empezar a trabajar por su perdición.
		\pend
		\endnumbering
	\end{Leftside}
	\selectlanguage{french}
	\begin{Rightside}
		\beginnumbering
		\pstart[\pagestyle{empty}\unexpanded{\section[Es que somos muy pobres/C'est qu'on est très pauvres]{C'est qu'on est très pauvres. Le Llano en flammes}}]
			Ici ça va de pire en pire. La semaine dernière tante Jacinta est morte, et samedi, alors qu’on avait fini de l’enterrer et que notre tristesse allait diminuant, il a commencé à pleuvoir comme jamais. Mon papa, ça l’a mis en colère, parce que toute la récolte d’orge prenait le soleil sur le terrain. Et l’averse est arrivée d’un coup, par grandes vagues d’eau, sans même nous laisser le temps de mettre à l’abri la moindre botte ; la seule chose que les nôtres aient pu faire c’est de rester blottis sous l’appentis, en regardant comment l’eau froide qui tombait du ciel brûlait cette orge jaune si fraîchement coupée.
		\pend
		\pstart
			Et c’est hier encore, quand ma sœur Tacha venait de fêter ses douze ans, qu’on a su que la vache que mon papa lui avait offerte pour sa fête avait été emportée par la rivière.
		\pend
		\pstart
			C’est il y a trois nuits que la rivière a commencé à croître, au petit matin. Moi, je dormais à poings fermés et, pourtant, le fracas de la rivière qui rampait ça m’a fait me réveiller en sursaut et bondir hors de mon lit couverture à la main, comme si j’avais cru que le toit de ma maison était en train de s’effondrer. Mais après je me suis rendormi, parce que j’ai reconnu le son de la rivière et parce que ce son s’est fait de plus en plus constant jusqu’à m’apporter à nouveau le sommeil.
		\pend
		\pstart
			Quand je me suis levé, la matinée était pleine de gros nuages et il avait dû pleuvoir sans s’arrêter. Ça se voyait à ce que le bruit de la rivière était plus fort et s’entendait plus proche. On sentait, comme on sent un incendie, l’odeur de pourri de l’eau agitée.
		\pend
		\pstart
			Lorsque j’ai voulu y jeter un œil, la rivière avait déjà quitté ses berges. Elle remontait petit à petit la calle real, et elle s’introduisait dans la maison de cette femme qu’on appelle \textit{la Tambora}. On entendait le clapotement de l’eau entrant par la basse-cour et sortant par grands jets par la porte. \textit{La Tambora} allait et venait marchant sur ce qui n’était déjà plus qu’un bout de rivière, chassant ses poules vers la rue pour qu’elles aillent se cacher quelque-part à l’abri du courant.
		\pend
		\pstart
			Et de l’autre côté, au niveau du tournant, la rivière avait dû emporter, qui sait depuis quand, le tamarin qui était sur la parcelle de ma tante Jacinta, puisque maintenant on n’y voit plus aucun tamarin. C’est que c’était le seul au village, c’est pour ça que les gens s’en rendent compte, que cette crue-là c’est la plus grande que la rivière nous ait donnée depuis des années.
		\pend
		\pstart
			On y est retournés l’après-midi, ma sœur et moi, pour regarder cet entassement d’eau, de plus en plus épaisse et foncée et qui passe déjà bien au-dessus d’où doit se trouver le pont. On y a passé des heures et des heures à voir cette chose-là sans se lasser. Après on est remontés par le ravin, parce qu’on voulait mieux entendre ce que les gens disaient, car en bas, auprès de la rivière, il y a un tel raffut et puis on voit seulement des bouches nombreuses qui s’ouvrent et se referment et c’est comme si elles voulaient dire quelque-chose ; mais on n’entend rien. C’est pour ça qu’on est remontés par le ravin, où il y a aussi des gens qui regardent la rivière et comptent les dommages qu’elle a causés. C’est là-bas qu’on a su que la rivière avait emporté \textit{la Serpentina}, cette vache qui était à ma sœur Tacha parce que mon papa la lui avait offerte le jour de son anniversaire et qui avait une oreille blanche et l’autre fauve et de très jolis yeux.
		\pend
		\pstart
			Je comprends toujours pas ce qui lui avait pris, à \textit{la Serpentina}, de traverser cette rivière, alors qu’elle savait bien que c’était pas là la rivière telle qu’elle avait l’habitude de la connaître. \textit{La Serpentina} n’était pourtant pas si sotte. Le plus probable c’est qu’elle avait dû se trouver endormie pour s’être laissé tuer comme ça sans rime ni raison. Moi, il m’arrivait souvent de devoir la réveiller quand je lui ouvrais la porte de l’enclos, parce que, sinon, d’elle-même, elle y serait restée toute la journée avec les yeux fermés, toute calme et à soupirer, comme on entend soupirer les vaches lorsqu’elles dorment.
		\pend
		\pstart
			Et là aussi elle devait dormir pareil. Peut-être qu’elle s’est quand même réveillée lorsqu’elle a senti que l’eau lourde lui cognait les côtes. Peut-être bien qu’alors elle a pris peur et elle a tenté de rentrer ; mais en se retournant elle s’est retrouvée embrouillée et affolée au milieu de tant d’eau noire et dure comme de la glaise. Peut-être qu’elle a meuglé appelant à l’aide.
		\pend
		\pstart
			Elle a meuglé, Dieu seul sait comment.
		\pend
		\pstart
			J’ai demandé à un monsieur qui a vu comment la rivière l’entraînait s’il n’avait pas aperçu aussi le petit veau qui traînait à ses côtés. Mais l’homme m’a dit qu’il ne savait pas s’il l’avait vu. Il m’a seulement dit que la vache tachetée est passée les quatre fers en l’air très près d’où il se trouvait et que là elle avait culbuté et qu’ensuite il n’a revu ni les cornes ni les fers ni le moindre autre signe d’une vache. Plein de troncs d’arbre roulaient dans la rivière avec les racines et tout et lui était très occupé à en tirer du bois, alors il ne pouvait pas remarquer si c’était des animaux ou des troncs ce qu’elle entraînait.
		\pend
		\pstart
			Voilà pourquoi on sait pas si le veau est vivant, ou s’il a suivi sa mère, emporté. Dans ce cas, que Dieu les protège tous les deux.
		\pend
		\pstart
			Ce qui les inquiète chez nous c’est ce qui pourrait se passer à l’avenir, maintenant que ma sœur Tacha est restée sans rien. Parce que c’est par un travail de longue haleine que mon papa avait obtenu \textit{la Serpentina}, alors qu’elle n’était qu’une vachette, pour la donner à ma sœur, afin de lui garantir un petit capital et qu’elle ne se mette pas à racoler comme l’ont fait mes deux autres sœurs, les aînées.
		\pend
		\pstart
			Selon mon papa, elles s’étaient débauchées parce que nous étions très pauvres à la maison et elles étaient très rebelles. Déjà petites elles étaient du genre désobéissant. Et dès qu’elles ont eu grandi elles se sont mises à fréquenter des hommes des plus mauvais, qui leur ont montré toutes ces mauvaises choses. Elles, elles ont tôt eu d’apprendre et comprenaient très bien les sifflets, quand on les appelait tard dans la nuit. Après elles sortaient même le jour. Elles partaient à tout instant chercher de l’eau à la rivière et parfois, quand on s’y attendait le moins, on les voyait dans la basse-cour, se vautrant sur le sol, à poil et un homme agrippé sur chacune d’entre elles.
		\pend
		\pstart
			Alors mon papa les a chassées toutes les deux. D’abord il les a supportées autant qu’il a pu ; mais plus tard, là, il n’a pas pu les supporter davantage et les a chassé hors de la maison. Elles, elles sont parties vers Ayutla ou je sais pas où ; mais elles courent les trottoirs.
		\pend
		\pstart
			C’est pour ça que c’est un crève-cœur pour mon papa, maintenant vis-à-vis de Tacha, qu’il ne veut pas voir imiter ses deux sœurs, en réalisant comme elle se retrouve si pauvre sans sa vache, en voyant qu’elle n’aura plus avec quoi se distraire jusqu’à ce qu’elle grandisse et puisse se marier avec un homme bon, qui puisse l’aimer pour toujours. Et ça aussi ça va être compliqué. Avec la vache c’était une autre histoire, car les prétendants n’auraient pas manqué, rien que pour remporter du même coup cette si jolie vache.
		\pend
		\pstart
			Le seul espoir qu’il nous reste c’est que le veau soit encore en vie. Pourvu qu’il n’ait pas tenté de traverser la rivière après sa mère. Parce qu’alors ma sœur Tacha est à ça de prendre le chemin de la rue. Et maman ne veut pas.
		\pend
		\pstart
			Ma maman ne sait pas pourquoi Dieu l’a tant punie en lui donnant des filles de ce genre-là, alors qu’il ne s’était pourtant jamais trouvé dans sa famille, de sa grand-mère jusqu’à nos jours, de gens malhonnêtes. Tous avaient été éduqués dans la peur de Dieu et étaient très obéissants et ne commettaient jamais d’irrévérences. Tous avaient été de ce style-là. Qui sait d’où ce mauvais exemple leur était venu, à ses deux filles. Elle ne s’en souvient pas. Elle ressasse tous ses souvenirs et ne voit vraiment pas de mal ou de péché à l’origine de la naissance l’une après l’autre de filles avec la même mauvaise habitude. Ça ne lui revient pas. Et à chaque fois qu’elle pense à elles, elle pleure et dit : \og{}Que Dieu les protège toutes les deux.\fg
		\pend
		\pstart
			Mais mon papa voit la chose différemment, c’est irrémédiable. Celle qui est dangereuse c’est celle qui nous reste, ici, la Tacha, qui grandit et grandit comme le tronc de l’ocote et qui a déjà des débuts de seins qui promettent d’être comme ceux de ses sœurs : pointus et hauts et comme espiègles pour attirer l’attention.
		\pend
		\pstart
			--- Oui --\,dit-il\,--, elle en mettra plein les yeux à qui la verra et où qu’elle se laisse voir. Et ça va mal finir ; c’est comme si je le voyais que ça va mal se finir.
			Voilà le crève-cœur de mon papa.
		\pend
		\pstart
			Et Tacha pleure de savoir que sa vache ne reviendra pas parce que la rivière la lui a tuée. Elle est là, à mes côtés, avec sa robe à la couleur des roses, regardant la rivière depuis le ravin et sans s’arrêter de pleurer. Sur son visage l’eau coule à flots, sale comme si la rivière s’était fourrée en elle.
		\pend
		\pstart
			Moi, je l’embrasse, j’essaye de la consoler, mais elle comprend pas. Elle n’en pleure que davantage. De sa bouche sort un bruit semblable à celui que la rivière traîne le long des berges, qui la fait trembler et la secoue toute entière, et, entre-temps, la crue continue de monter. La saveur de pourri qui vient de tout là-bas éclabousse le visage mouillé de Tacha et ses deux petits seins remuent de haut en bas, sans s’arrêter, comme si soudain il commençaient à gonfler pour concourir à sa perte.
		\pend
		\endnumbering
	\end{Rightside}

\end{pages}
\Pages
