\pstart[\pagestyle{empty}\unexpanded{\section[Es que somos muy pobres/C'est qu'on est très pauvres]{Es que somos muy pobres. El Llano en llamas}}]
	Aquí todo va de mal en peor. La semana pasada se murió mi tía Jacinta, y el sábado, cuando ya la habíamos enterrado y comenzaba a bajársenos la tristeza, comenzó a llover como nunca. A mi papá eso le dio coraje, porque toda la cosecha de cebada estaba asoleándose en el solar. Y el aguacero llegó de repente, en grandes olas de agua, sin darnos tiempo ni siquiera a esconder aunque fuera un manojo; lo único que pudimos hacer, todos los de mi casa, fue estarnos arrimados debajo del tejabán, viendo cómo el agua fría que caía del cielo quemaba aquella cebada amarilla tan recién cortada.
\pend
%
\pstart
	Y apenas ayer, cuando mi hermana Tacha acababa de cumplir doce años, supimos que la vaca que mi papá le regaló para el día de su santo se la había llevado el río.
\pend
%
\pstart
	El río comenzó a crecer hace tres noches, a eso de la madrugada. Yo estaba muy dormido y, sin embargo, el estruendo que traía el río al arrastrarse me hizo despertar en seguida y pegar el brinco de la cama con mi cobija en la mano, como si hubiera creído que se estaba derrumbando el techo de mi casa. Pero después me volví a dormir, porque reconocí el sonido del río y porque ese sonido se fue haciendo igual hasta traerme otra vez el sueño.
\pend
%
\pstart
	Cuando me levanté, la mañana estaba llena de nublazones y parecía que había seguido lloviendo sin parar. Se notaba en que el ruido del río era más fuerte y se oía más cerca. Se olía, como se huele una quemazón, el olor a podrido del agua revuelta.
\pend
%
\pstart
	A la hora en que me fui a asomar, el río ya había perdido sus orillas. Iba subiendo poco a poco por la calle real, y estaba metiéndose a toda prisa en la casa de esa mujer que le dicen \textit{la Tambora}. El chapaleo del agua se oía al entrar por el corral y al salir en grandes chorros por la puerta. \textit{La Tambora} iba y venía caminando por lo que era ya un pedazo de río, echando a la calle sus gallinas para que se fueran a esconder a algún lugar donde no les llegara la corriente.
\pend
%
\pstart
	Y por el otro lado, por donde está el recodo, el río se debía de haber llevado, quién sabe desde cuándo, el tamarindo que estaba en el solar de mi tía Jacinta, porque ahora ya no se ve ningún tamarindo. Era el único que había en el pueblo, y por eso nomás la gente se da cuenta de que la creciente esta que vemos es la más grande de todas las que ha bajado el río en muchos años.
\pend
%
\pstart
	Mi hermana y yo volvimos a ir por la tarde a mirar aquel amontonadero de agua que cada vez se hace más espesa y oscura y que pasa ya muy por encima de donde debe estar el puente. Allí nos estuvimos horas y horas sin cansarnos viendo la cosa aquella. Después nos subimos por la barranca, porque queríamos oír bien lo que decía la gente, pues abajo, junto al río, hay un gran ruidazal y sólo se ven las bocas de muchos que se abren y se cierran y como que quieren decir algo; pero no se oye nada. Por eso nos subimos por la barranca, donde también hay gente mirando el río y contando los perjuicios que ha hecho. Allí fue donde supimos que el río se había llevado a \textit{la Serpentina} la vaca esa que era de mi hermana Tacha porque mi papá se la regaló para el día de su cumpleaños y que tenía una oreja blanca y otra colorada y muy bonitos ojos.
\pend
%
\pstart
	No acabo de saber por qué se le ocurriría a La Serpentina pasar el río este, cuando sabía que no era el mismo río que ella conocía de a diario. \textit{La Serpentina} nunca fue tan atarantada. Lo más seguro es que ha de haber venido dormida para dejarse matar así nomás por nomás. A mí muchas veces me tocó despertarla cuando le abría la puerta del corral porque si no, de su cuenta, allí se hubiera estado el día entero con los ojos cerrados, bien quieta y suspirando, como se oye suspirar a las vacas cuando duermen.
\pend
%
\pstart
	Y aquí ha de haber sucedido eso de que se durmió. Tal vez se le ocurrió despertar al sentir que el agua pesada le golpeaba las costillas. Tal vez entonces se asustó y trató de regresar; pero al volverse se encontró entreverada y acalambrada entre aquella agua negra y dura como tierra corrediza. Tal vez bramó pidiendo que le ayudaran.
\pend
%
\pstart
	Bramó como sólo Dios sabe cómo.
\pend
%
\pstart
	Yo le pregunté a un señor que vio cuando la arrastraba el río si no había visto también al becerrito que andaba con ella. Pero el hombre dijo que no sabía si lo había visto. Sólo dijo que la vaca manchada pasó patas arriba muy cerquita de donde él estaba y que allí dio una voltereta y luego no volvió a ver ni los cuernos ni las patas ni ninguna señal de vaca. Por el río rodaban muchos troncos de árboles con todo y raíces y él estaba muy ocupado en sacar leña, de modo que no podía fijarse si eran animales o troncos los que arrastraba.
\pend
%
\pstart
	Nomás por eso, no sabemos si el becerro está vivo, o si se fue detrás de su madre
	río abajo. Si así fue, que Dios los ampare a los dos.
\pend
%
\pstart
	La apuración que tienen en mi casa es lo que pueda suceder el día de mañana, ahora que mi hermana Tacha se quedó sin nada. Porque mi papá con muchos trabajos había conseguido a \textit{la Serpentina}, desde que era una vaquilla, para dársela a mi hermana, con el fin de que ella tuviera un capitalito y no se fuera a ir de piruja como lo hicieron mis otras dos hermanas, las más grandes.
\pend
%
\pstart
	Según mi papá, ellas se habían echado a perder porque éramos muy pobres en mi casa y ellas eran muy retobadas. Desde chiquillas ya eran rezongonas. Y tan luego que crecieron les dio por andar con hombres de lo peor, que les enseñaron cosas malas. Ellas aprendieron pronto y entendían muy bien los chiflidos, cuando las llamaban a altas horas de la noche. Después salían hasta de día. Iban cada rato por agua al río y a veces, cuando uno menos se lo esperaba, allí estaban en el corral, revolcándose en el suelo, todas encueradas y cada una con un hombre trepado encima.
\pend
%
\pstart
	Entonces mi papá las corrió a las dos. Primero les aguantó todo lo que pudo; pero más tarde ya no pudo aguantarlas más y les dio carrera para la calle. Ellas se fueron para Ayutla o no sé para dónde; pero andan de pirujas.
\pend
%
\pstart
	Por eso le entra la mortificación a mi papá, ahora por la Tacha, que no quiere vaya a resultar como sus otras dos hermanas, al sentir que se quedó muy pobre viendo la falta de su vaca, viendo que ya no va a tener con qué entretenerse mientras le da por crecer y pueda casarse con un hombre bueno, que la pueda querer para siempre. Y eso ahora va a estar difícil. Con la vaca era distinto, pues no hubiera faltado quien se hiciera el ánimo de casarse con ella, sólo por llevarse también aquella vaca tan bonita.
\pend
%
\pstart
	La única esperanza que nos queda es que el becerro esté todavía vivo. Ojalá no se le haya ocurrido pasar el río detrás de su madre. Porque si así fue, mi hermana Tacha está tantito así de retirado de hacerse piruja. Y mamá no quiere.
\pend
%
\pstart
	Mi mamá no sabe por qué Dios la ha castigado tanto al darle unas hijas de ese modo, cuando en su familia, desde su abuela para acá, nunca ha habido gente mala. Todos fueron criados en el temor de Dios y eran muy obedientes y no le cometían irreverencias a nadie. Todos fueron por el estilo. Quién sabe de dónde les vendría a ese par de hijas suyas aquel mal ejemplo. Ella no se acuerda. Le da vueltas a todos sus recuerdos y no ve claro dónde estuvo su mal o el pecado de nacerle una hija tras otra con la misma mala costumbre. No se acuerda. Y cada vez que piensa en ellas, llora y dice: \og{}Que Dios las ampare a las dos.\fg{}
\pend
%
\pstart
	Pero mi papá alega que aquello ya no tiene remedio. La peligrosa es la que queda aquí, la Tacha, que va como palo de ocote crece y crece y que ya tiene unos comienzos de senos que prometen ser como los de sus hermanas: puntiagudos y altos y medio alborotados para llamar la atención.
\pend
%
\pstart
	--- Sí --dice--, le llenará los ojos a cualquiera dondequiera que la vean. Y acabará mal;
	como que estoy viendo que acabará mal. Ésa es la mortificación de mi papá.
\pend
%
\pstart
	Y Tacha llora al sentir que su vaca no volverá porque se la ha matado el río. Está aquí a mi lado, con su vestido color de rosa, mirando el río desde la barranca y sin dejar de llorar. Por su cara corren chorretes de agua sucia como si el río se hubiera metido dentro de ella.
\pend
%
\pstart
	Yo la abrazo tratando de consolarla, pero ella no entiende. Llora con más ganas. De su boca sale un ruido semejante al que se arrastra por las orillas del río, que la hace temblar y sacudirse todita, y, mientras, la creciente sigue subiendo. El sabor a podrido que viene de allá salpica la cara mojada de Tacha y los dos pechitos de ella se mueven de arriba abajo, sin parar, como si de repente comenzaran a hincharse para empezar a trabajar por su perdición.
\pend
