\pstart[\pagestyle{empty}\unexpanded{\section[Es que somos muy pobres/C'est qu'on est très pauvres]{C'est qu'on est très pauvres. Le Llano en flammes}}]
	Ici ça va de pire en pire. La semaine dernière tante Jacinta est morte, et samedi, alors qu’on avait fini de l’enterrer et que notre tristesse allait diminuant, il a commencé à pleuvoir comme jamais. Mon papa, ça l’a mis en colère, parce que toute la récolte d’orge prenait le soleil sur le terrain. Et l’averse est arrivée d’un coup, par grandes vagues d’eau, sans même nous laisser le temps de mettre à l’abri la moindre botte ; la seule chose que les nôtres aient pu faire c’est de rester blottis sous l’appentis, en regardant comment l’eau froide qui tombait du ciel brûlait cette orge jaune si fraîchement coupée.
\pend
%
\pstart
	Et tout juste hier, quand ma sœur Tacha venait de fêter ses douze ans, on a su que la vache que mon papa lui avait offerte pour sa fête avait été emportée par la rivière.
\pend
%
\pstart
	La rivière a commencé à croître il y a trois nuits, au petit matin. Moi, je dormais à poings fermés et, pourtant, le fracas de la rivière qui rampait m’a fait me réveiller en sursaut et bondir hors de mon lit couverture à la main, comme si j’avais cru que le toit de ma maison était en train de s’effondrer. Mais après je me suis rendormi, parce que j’ai reconnu le son de la rivière et parce que ce son s’est fait de plus en plus constant jusqu’à m’apporter à nouveau le sommeil.
\pend
%
\pstart
	Quand je me suis levé, la matinée était pleine de gros nuages et il avait dû pleuvoir sans s’arrêter. Ça se voyait à ce que le bruit de la rivière était plus fort et s’entendait plus proche. On sentait, comme on sent un incendie, l’odeur de pourri de l’eau troublée.
\pend
%
\pstart
	Lorsque j’ai voulu y jeter un œil, la rivière avait déjà quitté ses berges. Elle remontait petit à petit la calle real, et elle s’introduisait dans la maison de cette femme qu’on appelle \textit{la Tambora}. On entendait le clapotement de l’eau qui entrait par la basse-cour et resortait par grands jets par la porte. \textit{La Tambora} allait et venait marchant sur ce qui n’était déjà plus qu’un bout de rivière, chassant ses poules vers la rue pour qu’elles aillent se cacher quelque-part à l’abri du courant.
\pend
%
\pstart
	Et de l’autre côté, au niveau du tournant, la rivière avait dû emporter, qui sait depuis quand, le tamarin qui était sur la parcelle de ma tante Jacinta, puisque maintenant on n’y voit plus aucun tamarin. C’est que c’était le seul au village, c’est pour ça que les gens s’en rendent compte, que cette crue-là c’est la plus grande que la rivière nous ait donnée depuis des années.
\pend
%
\pstart
	On y est retournés l’après-midi, ma sœur et moi, pour regarder cet entassement d’eau, de plus en plus épaisse et foncée et qui passe déjà bien au-dessus d’où doit se trouver le pont. On y a passé des heures et des heures à voir cette chose-là sans se lasser. Après on est remontés par le ravin, parce qu’on voulait mieux entendre ce que les gens disaient, car en bas, au bord de la rivière, il y a un tel raffut et puis on voit seulement des bouches nombreuses qui s’ouvrent et se referment et c’est comme si elles voulaient dire quelque-chose ; mais on n’entend rien. C’est pour ça qu’on est remontés par le ravin, où il y a aussi des gens qui regardent la rivière et comptent les dommages qu’elle a causés. C’est là-bas qu’on a su que la rivière avait emporté \textit{la Serpentina}, cette vache qui était à ma sœur Tacha parce que mon papa la lui avait offerte le jour de son anniversaire et qui avait une oreille blanche et l’autre fauve et de très jolis yeux.
\pend
%
\pstart
	Je comprends toujours pas ce qui lui avait pris, à \textit{la Serpentina}, de traverser cette rivière, alors qu’elle savait bien que c’était pas là la rivière telle qu’elle avait l’habitude de la connaître. \textit{La Serpentina} n’était pourtant pas si sotte. Le plus probable c’est qu’elle avait dû se trouver endormie pour s’être laissé tuer comme ça sans rime ni raison. Moi, il m’arrivait souvent de devoir la réveiller quand je lui ouvrais la porte de l’enclos, parce que, sinon, d’elle-même, elle y serait restée toute la journée avec les yeux fermés, toute calme et à soupirer, comme on entend soupirer les vaches lorsqu’elles dorment.
\pend
%
\pstart
	Et là aussi elle devait dormir pareil. Peut-être qu’elle s’est quand même réveillée lorsqu’elle a senti que l’eau lourde lui cognait les côtes. Peut-être bien qu’alors elle a pris peur et elle a tenté de rentrer ; mais en se retournant elle s’est retrouvée embrouillée et affolée au milieu de tant d’eau noire et dure comme de la glaise. Peut-être qu’elle a meuglé appelant à l’aide.
\pend
%
\pstart
	Elle a meuglé, Dieu seul sait comment.
\pend
%
\pstart
	J’ai demandé à un monsieur qui a vu comment la rivière l’entraînait s’il n’avait pas aperçu aussi le petit veau qui traînait à ses côtés. Mais l’homme m’a dit qu’il ne savait pas s’il l’avait vu. Il m’a seulement dit que la vache tachetée est passée les quatre fers en l’air très près d’où il se trouvait et que là elle avait culbuté et qu’ensuite il n’a revu ni les cornes ni les fers ni le moindre autre signe d’une vache. Plein de troncs d’arbre roulaient dans la rivière avec les racines et tout et lui était très occupé à en tirer du bois, alors il ne pouvait pas remarquer si c’était des animaux ou des troncs ce qu’elle entraînait.
\pend
%
\pstart
	Voilà pourquoi on sait pas si le veau est vivant, ou s’il a suivi sa mère, emporté. Dans ce cas, que Dieu les protège tous les deux.
\pend
%
\pstart
	Ce qui les inquiète chez nous c’est ce qui pourrait se passer à l’avenir, maintenant que ma sœur Tacha est restée sans rien. Parce que c’est par un travail de longue haleine que mon papa avait obtenu \textit{la Serpentina}, alors qu’elle n’était qu’une vachette, pour la donner à ma sœur, afin de lui garantir un petit capital et qu’elle ne se mette pas à racoler comme l’ont fait mes deux autres sœurs, les aînées.
\pend
%
\pstart
	Selon mon papa, elles s’étaient débauchées parce que nous étions très pauvres à la maison et elles étaient très rebelles. Déjà petites elles étaient du genre désobéissant. Et dès qu’elles ont eu grandi elles se sont mises à fréquenter des hommes des plus mauvais, qui leur ont montré toutes ces mauvaises choses. Elles, elles ont tôt eu d’apprendre et comprenaient très bien les sifflets, quand on les appelait tard dans la nuit. Après elles sortaient même le jour. Elles partaient à tout instant chercher de l’eau à la rivière et parfois, quand on s’y attendait le moins, on les voyait dans la basse-cour, se vautrant sur le sol, à poil et un homme agrippé sur chacune d’entre elles.
\pend
%
\pstart
	Alors mon papa les a chassées toutes les deux. D’abord il les a supportées autant qu’il a pu ; mais plus tard, là, il n’a pas pu les supporter davantage et les a chassé hors de la maison. Elles, elles sont parties vers Ayutla ou je sais pas où ; mais elles courent les trottoirs.
\pend
%
\pstart
	C’est pour ça que c’est un crève-cœur pour mon papa, maintenant vis-à-vis de Tacha, qu’il ne veut pas voir imiter ses deux sœurs, en réalisant comme elle se retrouve si pauvre sans sa vache, en voyant qu’elle n’aura plus avec quoi se distraire jusqu’à ce qu’elle grandisse et puisse se marier avec un homme bon, qui puisse l’aimer pour toujours. Et ça aussi ça va être compliqué. Avec la vache c’était une autre histoire, car les prétendants n’auraient pas manqué, rien que pour remporter du même coup cette si jolie vache.
\pend
%
\pstart
	Le seul espoir qu’il nous reste c’est que le veau soit encore en vie. Pourvu qu’il n’ait pas tenté de traverser la rivière après sa mère. Parce qu’alors ma sœur Tacha est à ça de prendre le chemin de la rue. Et maman ne veut pas.
\pend
%
\pstart
	Ma maman ne sait pas pourquoi Dieu l’a tant punie en lui donnant des filles de ce genre-là, alors qu’il ne s’était pourtant jamais trouvé dans sa famille, de sa grand-mère jusqu’à nos jours, de gens malhonnêtes. Tous avaient été éduqués dans la peur de Dieu et étaient très obéissants et ne commettaient jamais d’irrévérences. Tous avaient été de ce style-là. Qui sait d’où ce mauvais exemple leur était venu, à ses deux filles. Elle ne s’en souvient pas. Elle ressasse tous ses souvenirs et ne voit vraiment pas de mal ou de péché à l’origine de la naissance l’une après l’autre de filles avec la même mauvaise habitude. Ça ne lui revient pas. Et à chaque fois qu’elle pense à elles, elle pleure et dit : \og{}Que Dieu les protège toutes les deux.\fg
\pend
%
\pstart
	Mais mon papa voit la chose différemment, c’est irrémédiable. Celle qui est dangereuse c’est celle qui nous reste, ici, la Tacha, qui grandit et grandit comme le tronc de l’ocote et qui a déjà des débuts de seins qui promettent d’être comme ceux de ses sœurs : pointus et hauts et comme espiègles pour attirer l’attention.
\pend
%
\pstart
	--- Oui --\,dit-il\,--, elle en mettra plein les yeux à qui la verra et où qu’elle se laisse voir. Et ça va mal finir ; c’est comme si je le voyais que ça va mal se finir.
	Voilà le crève-cœur de mon papa.
\pend
%
\pstart
	Et Tacha pleure de savoir que sa vache ne reviendra pas parce que la rivière la lui a tuée. Elle est là, à mes côtés, avec sa robe à la couleur des roses, regardant la rivière depuis le ravin et sans s’arrêter de pleurer. Sur son visage l’eau coule à flots, sale comme si la rivière s’était fourrée en elle.
\pend
%
\pstart
	Moi, je l’embrasse, j’essaye de la consoler, mais elle comprend pas. Elle n’en pleure que davantage. De sa bouche sort un bruit semblable à celui que la rivière traîne le long des berges, qui la fait trembler et la secoue toute entière, et, entre-temps, la crue continue de monter. La saveur de pourri qui vient de tout là-bas éclabousse le visage mouillé de Tacha et ses deux petits seins remuent de haut en bas, sans s’arrêter, comme si soudain il commençaient à gonfler pour concourir à sa perte.
\pend
