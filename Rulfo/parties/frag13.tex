\fancyhead[CE]{Fragmento \esno 13}
\fancyhead[CO]{Fragment \frno 13}

\begin{pages}
	\selectlanguage{spanish}
	\begin{Leftside}
		\beginnumbering

		\pstart[\pagestyle{empty}\unexpanded{\section[Fragmento 13/Fragment 13]{Fragmento \esno 13. Pedro Páramo}}]
			\guillemotleft Hay aire y sol, hay nubes. Allá arriba un cielo azul y detrás de él tal vez haya canciones; tal vez mejores voces\ldots Hay esperanza en suma. Hay esperanza para nosotros, contra nuestro pesar.
		\pend
		\pstart
			\guillemotright Pero no para ti, Miguel Páramo, que has muerto sin perdón y no alcanzarás ninguna gracia.\guillemotright
		\pend
		\pstart
			El padre Rentería dio vuelta al cuerpo y entregó la misa al pasado. Se dio prisa por terminar pronto y salió sin dar la bendición final a aquella gente que llenaba la iglesia.

			--- ¡Padre, queremos que nos lo bendiga!

			--- ¡No! --dijo moviendo negativamente la cabeza--. No lo haré. Fue un mal hombre y no entrará al Reino de los Cielos. Dios me tomará a mal que interceda por él.

		\pend
		\pstart
			Lo decía, mientras trataba de retener sus manos para que no enseñaran su temblor. Pero fue.

		\pend
		\pstart
			Aquel cadáver pesaba mucho en el ánimo de todos. Estaba sobre una tarima, en medio de la iglesia, rodeado de cirios nuevos, de flores, de un padre que estaba detrás de él, solo, esperando que terminara la velación.
		\pend
		\pstart
			El padre Rentería pasó junto a Pedro Páramo procurando no rozarle los hombros. Levantó el hisopo con ademanes suaves y roció el agua bendita de arriba abajo, mientras salía de su boca un murmullo, que podía ser de oraciones. Después se arrodilló y todo el mundo se arrodilló con él:

			--- Ten piedad de tu siervo, Señor.

			--- Que descanse en paz, amén --contestaron las voces.
		\pend
		\pstart
			Y cuando empezaba a llenarse nuevamente de cólera, vio que todos abandonaban la iglesia llevándose el cadáver de Miguel Páramo.
		\pend
		\pstart
			Pedro Páramo se acercó, arrodillándose a su lado:

			--- Yo sé que usted lo odiaba, padre. Y con razón. El asesinato de su hermano, que según rumores fue cometido por mi hijo; el caso de su sobrina Ana, violada por él según el juicio de usted; las ofensas y falta de respeto que le tuvo en ocasiones, son motivos que cualquiera puede admitir. Pero olvídese ahora, padre. Considérelo y perdónelo como quizá Dios lo haya perdonado.
		\pend
		\pstart
			Puso sobre el reclinatorio un puño de monedas de oro y se levantó:

			--- Reciba eso como una limosna para su iglesia.
		\pend
		\pstart
			La iglesia estaba ya vacía. Dos hombres esperaban en la puerta de Pedro Páramo, quien se juntó con ellos, y juntos siguieron el féretro que aguardaba descansando sobre los hombros de cuatro caporales de la Media Luna.
		\pend
		\pstart
			El padre Rentería recogió las monedas una por una y se acercó al altar.

			--- Son tuyas --dijo--. Él puede comprar la salvación. Tú sabes si éste es el precio. En cuanto a mí, Señor, me pongo ante tus plantas para pedirte lo justo o lo injusto, que todo nos es dado pedir\ldots Por mí, condénalo, Señor.
		\pend
		\pstart
			Y cerró el sagrario.
		\pend
		\pstart
			Entró en la sacristía, se echó en un rincón, y allí lloró de pena y de tristeza
			hasta agotar sus lágrimas.

			--- Está bien, Señor, tú ganas --dijo después.
		\pend
		\endnumbering
	\end{Leftside}
	\selectlanguage{french}
	\begin{Rightside}
		\beginnumbering
		\pstart[\pagestyle{empty}\unexpanded{\section[Fragmento 13/Fragment 13]{Fragment \frno 13. Pédro Paramo}}]
			\og{}Il y a de l'air et du soleil, il y a des nuages. Là-haut un ciel azur et derrière lui peut-être aussi des chansons ; peut-être de meilleures voix\ldots Il y a de l'espoir, en somme. Il y a de l'espoir pour nous, remède à notre pénitence.
		\pend
		\pstart
			\fg{}Mais pas pour toi, Miguel Paramo, tu es mort sans pardon et n'obtiendras aucune grâce.\fg{}
		\pend
		\pstart
			Le père Rentéria se détourna du corps et livra la messe au passé. Il se dépêcha de finir tôt et sortit sans prononcer la bénédiction finale à ces gens qui remplissaient l'église.

			--- Père, on veut que vous nous le bénissiez !

			--- Non ! --\,dit-il en hochant la tête négativement\,--. Je ne le ferai pas. Ce fut un homme mauvais et il n'entrera pas au Royaume des Cieux. Dieu m'en voudrait d'intercéder en sa faveur.
		\pend
		\pstart
			Ce disant, il tentait de retenir ses mains pour qu'elles ne montrent pas leur tremblement. Sans succès.
		\pend
		\pstart
			Ce cadavre pesait beaucoup sur l'humeur de tous. Il était sur une estrade, au milieu de l'église, entouré de cierges neufs, de fleurs, d'un père qui était derrière lui, seul, en attendant la fin de la veillée.
		\pend
		\pstart
			Le père passa à côté de Pédro Paramo veillant à ne pas lui frôler les épaules. Il leva le goupillon d'un geste suave et aspergea de l'eau bénite de haut en bas, tandis qu'un murmure sortait de sa bouche, vraisemblablement des prières. Ensuite, il s'agenouilla et tout le monde s'agenouilla à son tour :

			--- Aie pitié de ton serviteur, Seigneur.

			--- Repose en paix, amen --\,répondirent les voix.
		\pend
		\pstart
			Et alors qu'il commençait nouvellement à bouillir de colère, il vit que tous abandonnaient l'église emportant le cadavre de Miguel Paramo.
		\pend
		\pstart
			Pédro Paramo s'approcha, s'agenouillant à ses côtés :

			--- Je sais que vous le haïssiez, mon père. Et à raison. L'assassinat de votre frère, qui selon les rumeurs fut commis par mon fils ; le cas de votre nièce Ana, violée par lui selon votre jugement ; les offenses et le manque de respect dont il a quelquefois fait preuve à votre endroit, sont des raisons que l'on admet aisément. Mais oubliez à présent, père. Reconsidérez-le et pardonnez-lui comme Dieu peut-être lui a déjà pardonné.
		\pend
		\pstart
			Il posa sur le prie-Dieu une poignée de pièces d'or et se leva :

			--- Acceptez cela comme une aumône pour votre église.
		\pend
		\pstart
			L'église était déjà vide. À la porte, deux hommes attendaient Pédro Paramo, qui se joignit à eux, et ensemble ils suivirent le cercueil qui attendait reposé sur les épaules de quatre contre-maîtres de la Media Luna.
		\pend
		\pstart
			Le père Rentéria ramassa les pièces de monnaie une à une et s'approcha de l'autel.

			--- Elles sont à toi --\,dit-il\,--. Il peut acheter le salut. À toi de voir si c'en est le prix. Pour ma part, Seigneur, je me prosterne à tes pieds pour te demander ce qui est juste et ce qui ne l'est pas, puisqu'il nous est donné de tout demander\ldots Pour moi, tu peux le condamner, Seigneur.
		\pend
		\pstart
			Et il referma le tabernacle.
		\pend
		\pstart
			Il entra dans la sacristie, s'affala dans un coin, et y pleura de peine et de tristesse jusqu'à écouler ses larmes.

			--- C'est bon, Seigneur, tu as gagné --\,dit-il ensuite.
		\pend
		\endnumbering
	\end{Rightside}

\end{pages}
\Pages
