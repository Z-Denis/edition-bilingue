\pstart[\pagestyle{empty}\unexpanded{\section[Fragmento 38/Fragment 38]{Fragment \frno 38. Pédro Paramo}}]
  --- Le temps doit être changeant, là dehors. Ma mère me disait que, dès qu’il commençait à pleuvoir, tout se remplissait de lumières et de l’odeur verte des bourgeons. Elle me racontait comment arrivait la marée des nuages, comment ils se jetaient sur la terre et la décomposaient en altérant ses couleurs\ldots Ma mère, qui a passé son enfance et ses plus beaux jours dans ce village et qui n’a même pas pu venir mourir ici. Même pour ça, elle m’a envoyé moi, à sa place. C’est curieux, Dorotéa, comme je ne suis pas même parvenu à en voir le ciel. Au moins, peut-être, cela doit être le même que celui qu’elle a connu.
\pend
%
\pstart
  --- Je ne sais pas, Juan Préciado. Cela fait tant d’années que je ne levais pas le visage que j’en ai oublié le ciel. Et quand bien même je l'aurais fait, qu’y aurais-je gagné ? Le ciel est si haut, et mes yeux si dépourvus de regard, que savoir où tombait la terre suffisait à mon bonheur. En plus, j’ai perdu tout intérêt pour lui depuis que le père Rentéria m’a assuré que jamais je ne connaîtrais la Gloire. Que, même de loin, je ne la verrais pas\ldots C’est l’affaire de mes péchés ; mais, lui, il n’aurait pas dû me le dire. La vie est déjà une épreuve qui se suffit à elle-même. La seule chose qui pousse à avancer, c’est l’espoir d’être transporté d’un côté à l’autre après la mort ; mais quand une porte nous est fermée et celle qui reste ouverte n'est autre que celle de l'Enfer, alors mieux vaut ne jamais être né\ldots Le Ciel pour moi, Juan Préciado, est ici où je me trouve maintenant.
\pend
%
\pstart
  --- Et ton âme ? Où est-ce que tu crois qu’elle a bien pu aller ?

  --- Elle doit errer sur terre çà et là comme tant d’autres ; à la recherche de vivants qui prient pour elle. Il se peut qu’elle me haïsse pour les mauvais traitements que je lui ai infligés ; mais ça ne m’inquiète plus. Je me repose du vice de ses remords. Même le peu que je mangeais, elle me le rendait amer, et mes nuits, insupportables, les remplissant de pensées inquiétantes avec des figures de condamnés et des choses dans le genre. Quand je me suis assise pour mourir, elle m’a priée de me relever et de continuer à traîner ma vie, comme si elle attendait encore un miracle qui me lave de mes fautes. Je n’ai même pas essayé : \og{}Mon chemin s’arrête ici --\,lui ai-je dit\,--. Je n'ai plus la force de continuer.\fg{} Et j’ai ouvert la bouche pour la laisser partir. Et elle est partie. J’ai senti comme tombait sur mes mains le filet de sang qui la retenait attachée à mon cœur. %’amarrait à mon cœur.
  %J’ai senti comme tombait sur mes mains le filet de sang qui la maintenait arrimée à mon cœur.
\pend
