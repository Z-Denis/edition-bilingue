\pstart[\pagestyle{empty}\unexpanded{\section[Fragmento 38/Fragment 38]{Fragmento \esno 38. Pedro Páramo}}]
  --- Allá afuera debe estar variando el tiempo. Mi madre me decía que, en cuanto comenzaba a llover, todo se llenaba de luces y del olor verde de los retoños. Me contaba cómo llegaba la marea de las nubes, cómo se echaban sobre la tierra y la descomponían cambiándole los colores\ldots Mi madre, que vivió su infancia y sus mejores años en este pueblo y que ni siquiera pudo venir a morir aquí. Hasta para eso me mandó a mí en su lugar. Es curioso, Dorotea, cómo no alcancé a ver ni el cielo. Al menos, quizá, debe ser el mismo que ella conoció.
\pend
%
\pstart
  --- No lo sé, Juan Preciado. Hacía tantos años que no alzaba la cara, que me olvidé del cielo. Y aunque lo hubiera hecho, ¿qué habría ganado? El cielo está tan alto, y mis ojos tan sin mirada, que vivía contenta con saber dónde quedaba la tierra. Además, le perdí todo mi interés desde que el padre Rentería me aseguró que jamás conocería la gloria. Que ni siquiera de lejos la vería\ldots Fue cosa de mis pecados; pero él no debía habérmelo dicho. Ya de por sí la vida se lleva con trabajos. Lo único que la hace a una mover los pies es la esperanza de que al morir la lleven a una de un lugar a otro; pero cuando a una le cierran una puerta y la que queda abierta es nomás la del infierno, más vale no haber nacido\ldots El cielo para mí, Juan Preciado, está aquí donde estoy ahora.
\pend
%
\pstart
  --- ¿Y tu alma? ¿Dónde crees que haya ido?

  --- Debe andar vagando por la tierra como tantas otras; buscando vivos que recen por ella. Tal vez me odie por el mal trato que le di; pero eso ya no me preocupa. He descansado del vicio de sus remordimientos. Me amargaba hasta lo poco que comía, y me hacía insoportables las noches llenándomelas de pensamientos intranquilos con figuras de condenados y cosas de ésas. Cuando me senté a morir, ella rogó que me levantara y que siguiera arrastrando la vida, como si esperara todavía algún milagro que me limpiara de culpas. Ni siquiera hice el intento: \guillemotleft Aquí se acaba el camino --le dije--. Ya no me quedan fuerzas para más\guillemotright. Y abrí la boca para que se fuera. Y se fue. Sentí cuando cayó en mis manos el hilito de sangre con que estaba amarrada a mi corazón.
\pend
