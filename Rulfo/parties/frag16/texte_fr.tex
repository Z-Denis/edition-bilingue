\pstart[\pagestyle{empty}\unexpanded{\section[Fragmento 16/Fragment 16]{Fragment \frno 16. Pédro Paramo}}]
  Il y avait des étoiles filantes. À Comala, les lumières s'éteignirent.
\pend
%
\pstart
  Alors le ciel prit possession de la nuit.
\pend
%
\pstart
  Le père Rentéria se retournait dans son lit sans pouvoir dormir :

  \og{} Tout ce qui se passe est de ma faute --\,se dit-il\,--. La crainte d’offenser ceux qui me soutiennent. Parce que voilà la vérité ; c'est eux qui m’entretiennent. Des pauvres je n’obtiens rien ; les prières ne nourrissent pas son homme. Cela a toujours été comme ça. Et voilà les conséquences. Ma faute. J’ai trahi ceux qui m’aiment et qui m’ont donné leur foi et viennent me chercher, moi, pour que j’intercède pour eux auprès de Dieu. Mais qu’ont-ils obtenu avec leur foi ? Une place au Ciel ? Ou la purification de leurs âmes ? Et à quoi bon purifier son âme, si au dernier moment\ldots J’ai encore devant les yeux le regard de Maria Dyada, qui est venu me demander de sauver sa sœur Éduviges :
\pend
%
\pstart
  \guillemotright --- Elle a toujours été au service de ses semblables. Elle leur a donné tout ce qu’elle avait. Leur a même donné un fils, à tous. Et le leur a mis devant les yeux pour que quelqu’un le reconnaisse comme sien ; mais personne n’a voulu le faire. Alors elle leur a dit : ``Dans ce cas je suis également son père, quoique par hasard j’en aie été la mère.'' Ils ont abusé de son hospitalité par cette bonté qu’elle avait à ne pas vouloir offenser ni se mettre personne à dos.
\pend
%
\pstart
  \guillemotright --- Mais elle s’est suicidée. Elle a œuvré contre la main de Dieu.
\pend
%
\pstart
  \guillemotright --- Il ne lui restait pas d’autre issue. Elle s’y est résolue aussi par bonté.
\pend
%
\pstart
  \guillemotright --- Elle a failli à la dernière minute --\,voilà ce que je lui ai dit\,--. Au dernier moment. Tant de vertu accumulée pour son salut, et la perdre comme ça, d'un coup !
\pend
%
\pstart
  \guillemotright --- Mais puisqu’elle ne l’a pas perdue. Elle est morte dans d’atroces souffrances. Et la souffrance\ldots Vous nous avez dit quelque-chose à propos de la souffrance que je ne me rappelle plus. Elle, elle est partie par cette souffrance. Morte tordue par le sang qui la noyait. Je vois encore ses grimaces, et ses grimaces étaient les gestes les plus tristes qu’un être humain ait jamais faits.
\pend
%
\pstart
  \guillemotright --- Peut-être en redoublant de prières.
\pend
%
\pstart
  \guillemotright --- On ne fait que ça, père.
\pend
%
\pstart
  \guillemotright --- Je dirais bien peut-être, éventuellement, avec les messes grégoriennes ; mais pour cela on doit demander de l’aide, faire venir des prêtres. Et cela coûte de l'argent.
\pend
%
\pstart
  \guillemotright Là, face à mes yeux, se trouvait le regard de Maria Dyada, une pauvre femme pleine d’enfants.
\pend
%
\pstart
  \guillemotright --- Je n’ai pas cet argent. Vous le savez bien, père.
\pend
%
\pstart
  \guillemotright --- Laissons les choses comme elles sont. Comptons sur Dieu.
\pend
%
\pstart
  \guillemotright --- Oui, père.\fg{}
\pend
%
\pstart
  Pourquoi ce regard s’armait-il de courage face à la résignation ? Que lui coûtait-il, à lui, de pardonner, alors qu’il était si simple de dire un mot ou deux, ou cent mots si cela s’avérait nécessaire pour sauver une âme. Que savait-il, lui, du Ciel et de l’Enfer ? Et pourtant, lui, perdu au milieu d’un village sans nom, savait ceux qui avaient mérité le Ciel. Il y avait un catalogue. Il commença à parcourir les saints du panthéon catholique en commençant par ceux du jour : \og{} Sainte Nunilone, vierge et martyre ; Abercius, évêque ; les Saintes Salomé, veuve, Alodie ou Élodie et Nunilo, vierges ; Cordule et Donat. \fg{}Et il continuat. Le sommeil commençait déjà à le dominer lorsqu’il s’assit sur le lit : \og{} Je révise une suite de saints comme je compterais les moutons.\fg{}
\pend
%
\pstart
  Il sortit et regarda le ciel. Il pleuvait des étoiles. Il le regretta parce qu’il aurait voulu voir un ciel immobile. Il entendit les coqs chanter. Il sentit l’enveloppe de la nuit recouvrant la terre. La terre, \og{}cette vallée de larmes\fg{}.
\pend
