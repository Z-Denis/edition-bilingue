\pstart[\pagestyle{empty}\unexpanded{\section[Fragmento 16/Fragment 16]{Fragmento \esno 16. Pedro Páramo}}]
  Había estrellas fugaces. Las luces en Comala se apagaron.
\pend
%
\pstart
  Entonces el cielo se adueñó de la noche.
\pend
%
\pstart
  El padre Rentería se revolcaba en su cama sin poder dormir:

  \guillemotleft Todo esto que sucede es por mi culpa --se dijo--. El temor de ofender a quienes me sostienen. Porque ésta es la verdad; ellos me dan mi mantenimiento. De los pobres no consigo nada; las oraciones no llenan el estómago. Así ha sido hasta ahora. Y éstas son las consecuencias. Mi culpa. He traicionado a aquellos que me quieren y que me han dado su fe y me buscan para que yo interceda por ellos para con Dios. ¿Pero qué han logrado con su fe? ¿La ganancia del cielo? ¿O la purificación de sus almas? Y para qué purifican su alma, si en el último momento\ldots Todavía tengo frente a mis ojos la mirada de María Dyada, que vino a pedirme salvara a su hermana Eduviges:
\pend
%
\pstart
  \guillemotright --- Ella sirvió siempre a sus semejantes. Les dio todo lo que tuvo. Hasta les dio un hijo, a todos. Y se los puso enfrente para que alguien lo reconociera como suyo; pero nadie lo quiso hacer. Entonces les dijo: ``En ese caso yo soy también su padre, aunque por casualidad haya sido su madre.'' Abusaron de su hospitalidad por esa bondad suya de no querer ofenderlos ni de malquistarse con ninguno.
\pend
%
\pstart
  \guillemotright --- Pero ella se suicidó. Obró contra la mano de Dios.
\pend
%
\pstart
  \guillemotright --- No le quedaba otro camino. Se resolvió a eso también por bondad.
\pend
%
\pstart
  \guillemotright --- Falló a última hora --eso es lo que le dije--. En el último momento.
  ¡Tantos bienes acumulados para su salvación, y perderlos así de pronto!
\pend
%
\pstart
  \guillemotright --- Pero si no los perdió. Murió con muchos dolores. Y el dolor\ldots Usted nos ha dicho algo acerca del dolor que ya no recuerdo. Ella se fue por ese dolor. Murió retorcida por la sangre que la ahogaba. Todavía veo sus muecas, y sus muecas eran los más tristes gestos que ha hecho un ser humano.
\pend
%
\pstart
  \guillemotright --- Tal vez rezando mucho.
\pend
%
\pstart
  \guillemotright --- Vamos rezando mucho, padre.
\pend
%
\pstart
  \guillemotright --- Digo tal vez, si acaso, con las misas gregorianas; pero para eso necesitamos pedir ayuda, mandar traer sacerdotes. Y eso cuesta dinero.
\pend
\pstart
  \guillemotright Allí estaba frente a mis ojos la mirada de María Dyada, una pobre mujer llena de hijos.
\pend
%
\pstart
  \guillemotright --- No tengo dinero. Eso lo sabe, padre.
\pend
%
\pstart
  \guillemotright --- Dejemos las cosas como están. Esperemos en Dios.
\pend
%
\pstart
  \guillemotright --- Sí, padre.\guillemotright
\pend
%
\pstart
  ¿Por qué aquella mirada se volvía valiente ante la resignación? Qué le costaba a él perdonar, cuando era tan fácil decir una palabra o dos, o cien palabras si éstas fueran necesarias para salvar el alma. ¿Qué sabía él del cielo y del infierno? Y sin embargo, él, perdido en un pueblo sin nombre, sabía los que habían merecido el cielo. Había un catálogo. Comenzó a recorrer los santos del panteón católico comenzando por los del día: \guillemotleft Santa Nunilona, virgen y mártir; Abercio, obispo; santas Salomé viuda, Alodia o Elodia y Nulina, vírgenes; Córdula y Donato.\guillemotright Y siguió. Ya iba siendo dominado por el sueño cuando se sentó en la cama: \guillemotleft Estoy repasando una hilera de santos como si estuviera viendo saltar cabras.\guillemotright
\pend
%
\pstart
  Salió fuera y miró el cielo. Llovían estrellas. Lamentó aquello porque
  hubiera querido ver un cielo quieto. Oyó el canto de los gallos. Sintió la
  envoltura de la noche cubriendo la tierra. La tierra, \guillemotleft este valle de lágrimas\guillemotright.
\pend
