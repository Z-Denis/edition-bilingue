\pstart[\pagestyle{empty}\unexpanded{\section[Fragmento 13/Fragment 13]{Fragment \frno 13. Pédro Paramo}}]
  \og{}Il y a l'air et du soleil, il y a des nuages. Là-haut un ciel azur et derrière lui peut-être aussi des chansons ; peut-être de meilleures voix\ldots Il y a de l'espoir, en somme. Il y a de l'espoir pour nous, remède à notre pénitence.
\pend
%
\pstart
  \guillemotright Mais pas pour toi, Miguel Paramo, tu es mort sans pardon et n'obtiendras aucune grâce.\fg{}
\pend
%
\pstart
  Le père Rentéria se détourna du corps et livra la messe au passé. Il se dépêcha de finir tôt et sortit sans prononcer la bénédiction finale à ces gens qui remplissaient l'église.

  --- Père, il faut que vous nous le bénissiez !

  --- Non ! --\,dit-il en hochant la tête négativement\,--. Je ne le ferai pas. Ce fut un homme mauvais et il n'entrera pas au Royaume des Cieux. Dieu m'en voudrait d'intercéder en sa faveur.
\pend
%
\pstart
  Ce disant, il tentait de retenir ses mains pour qu'elles ne montrent pas leur tremblement. Sans succès.
\pend
%
\pstart
  Ce cadavre-là pesait beaucoup sur l'humeur de tous. Il était sur une estrade, au milieu de l'église, entouré de cierges neufs, de fleurs, d'un père qui se trouvait derrière lui, seul, en attendant la fin de la veillée.
\pend
%
\pstart
  Le père Renteria passa à côté de Pédro Paramo veillant à ne pas lui frôler les épaules. Il leva le goupillon d'un geste suave et en aspergea l'eau bénite de haut en bas, tandis qu'un murmure sortait de sa bouche, vraisemblablement des prières. Ensuite, il s'agenouilla et tout le monde s'agenouilla à son tour :

  --- Aie pitié de ton serviteur, Seigneur.

  --- Repose en paix, amen --\,répondirent les voix.
\pend
%
\pstart
  Et alors qu'il commençait nouvellement à bouillir de colère, il vit que tous abandonnaient l'église emportant le cadavre de Miguel Paramo.
\pend
%
\pstart
  Pédro Paramo s'approcha, s'agenouillant à ses côtés :

  --- Je sais que vous le haïssiez, mon père. Et à raison. L'assassinat de votre frère, qui selon les rumeurs fut commis par mon fils ; le cas de votre nièce Ana, violée par lui selon votre jugement ; les offenses et le manque de respect dont il a quelquefois fait preuve à votre endroit, sont des raisons que l'on admet aisément. Mais oubliez à présent, père. Reconsidérez-le et pardonnez-lui comme Dieu peut-être lui a déjà pardonné.
\pend
%
\pstart
  Il posa sur le prie-Dieu une poignée de pièces d'or et se leva :

  --- Acceptez cela comme une aumône pour votre église.
\pend
%
\pstart
  L'église était déjà vide. À la porte, deux hommes attendaient Pédro Paramo, qui se joignit à eux, et ensemble ils suivirent le cercueil qui attendait reposé sur les épaules de quatre contre-maîtres de la Media Luna.
\pend
%
\pstart
  Le père Rentéria ramassa les pièces de monnaie une à une et s'approcha de l'autel.

  --- Elles sont à toi --\,dit-il\,--. Il peut acheter le salut. À toi de voir si c'en est le prix. Pour ma part, Seigneur, je me prosterne à tes pieds pour te demander ce qui est juste et ce qui ne l'est pas, puisqu'il nous est donné de tout demander\ldots Pour moi, tu peux le condamner, Seigneur.
\pend
%
\pstart
  Et il referma le tabernacle.
\pend
%
\pstart
  Il entra dans la sacristie, s'affala dans un coin, et y pleura de peine et de tristesse jusqu'à écouler ses larmes.

  --- C'est bon, Seigneur, tu as gagné --\,dit-il ensuite.
\pend
