\pstart[\pagestyle{empty}\unexpanded{\section[Fragmento 13/Fragment 13]{Fragmento \esno 13. Pedro Páramo}}]
  \guillemotleft Hay aire y sol, hay nubes. Allá arriba un cielo azul y detrás de él tal vez haya canciones; tal vez mejores voces\ldots Hay esperanza en suma. Hay esperanza para nosotros, contra nuestro pesar.
\pend
%
\pstart
  \guillemotright Pero no para ti, Miguel Páramo, que has muerto sin perdón y no alcanzarás ninguna gracia.\guillemotright
\pend
%
\pstart
  El padre Rentería dio vuelta al cuerpo y entregó la misa al pasado. Se dio prisa por terminar pronto y salió sin dar la bendición final a aquella gente que llenaba la iglesia.

  --- ¡Padre, queremos que nos lo bendiga!

  --- ¡No! --dijo moviendo negativamente la cabeza--. No lo haré. Fue un mal hombre y no entrará al Reino de los Cielos. Dios me tomará a mal que interceda por él.
\pend
%
\pstart
  Lo decía, mientras trataba de retener sus manos para que no enseñaran su temblor. Pero fue.
\pend
%
\pstart
  Aquel cadáver pesaba mucho en el ánimo de todos. Estaba sobre una tarima, en medio de la iglesia, rodeado de cirios nuevos, de flores, de un padre que estaba detrás de él, solo, esperando que terminara la velación.
\pend
%
\pstart
  El padre Rentería pasó junto a Pedro Páramo procurando no rozarle los hombros. Levantó el hisopo con ademanes suaves y roció el agua bendita de arriba abajo, mientras salía de su boca un murmullo, que podía ser de oraciones. Después se arrodilló y todo el mundo se arrodilló con él:

  --- Ten piedad de tu siervo, Señor.

  --- Que descanse en paz, amén --contestaron las voces.
\pend
%
\pstart
  Y cuando empezaba a llenarse nuevamente de cólera, vio que todos abandonaban la iglesia llevándose el cadáver de Miguel Páramo.
\pend
%
\pstart
  Pedro Páramo se acercó, arrodillándose a su lado:

  --- Yo sé que usted lo odiaba, padre. Y con razón. El asesinato de su hermano, que según rumores fue cometido por mi hijo; el caso de su sobrina Ana, violada por él según el juicio de usted; las ofensas y falta de respeto que le tuvo en ocasiones, son motivos que cualquiera puede admitir. Pero olvídese ahora, padre. Considérelo y perdónelo como quizá Dios lo haya perdonado.
\pend
%
\pstart
  Puso sobre el reclinatorio un puño de monedas de oro y se levantó:

  --- Reciba eso como una limosna para su iglesia.
\pend
%
\pstart
  La iglesia estaba ya vacía. Dos hombres esperaban en la puerta de Pedro Páramo, quien se juntó con ellos, y juntos siguieron el féretro que aguardaba descansando sobre los hombros de cuatro caporales de la Media Luna.
\pend
%
\pstart
  El padre Rentería recogió las monedas una por una y se acercó al altar.

  --- Son tuyas --dijo--. Él puede comprar la salvación. Tú sabes si éste es el precio. En cuanto a mí, Señor, me pongo ante tus plantas para pedirte lo justo o lo injusto, que todo nos es dado pedir\ldots Por mí, condénalo, Señor.
\pend
%
\pstart
  Y cerró el sagrario.
\pend
%
\pstart
  Entró en la sacristía, se echó en un rincón, y allí lloró de pena y de tristeza
  hasta agotar sus lágrimas.

  --- Está bien, Señor, tú ganas --dijo después.
\pend
